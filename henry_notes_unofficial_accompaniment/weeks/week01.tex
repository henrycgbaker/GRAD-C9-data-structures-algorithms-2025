% Week 1: Introduction to Scrum and Agile Development
% Expanded from original lecture notes on coordinated systems and software development methodology

\thispagestyle{empty}
\begin{tabular}{p{15.5cm}}
{\large \bf DSA Lecture Notes 2024 \\ Henry Baker} \\
\hline
\end{tabular}

\vspace*{0.3cm}
\begin{center}
	{\Large \bf DSA Lecture Notes: Week 1\\ Introduction to Scrum and Agile Development}
	\vspace{2mm}
\end{center}
\vspace{0.4cm}

This introductory week establishes a foundational metaphor for understanding complex, adaptive systems-both biological and organisational. We begin with W. Grey Walter's pioneering work on cybernetic tortoises, which demonstrated how simple feedback loops can produce surprisingly sophisticated emergent behaviour. This concept directly informs our understanding of Scrum: a framework that harnesses complexity through iterative feedback rather than attempting to control it through rigid, top-down planning.

\section{Emergent Behaviour: Grey Walter's Tortoises}

W. Grey Walter (1910--1977) was a British neurophysiologist and roboticist whose work profoundly influenced both neuroscience and artificial intelligence. His 1953 book \textit{The Living Brain} documented experiments with simple autonomous robots that exhibited remarkably complex behaviour.

\subsection{The Machina Speculatrix}

In the early 1950s, Walter constructed a series of three-wheeled tortoise-shaped robots he called \textit{machina speculatrix} (literally ``the machine that watches''). These robots-nicknamed Elmer and Elsie-were groundbreaking in their simplicity and the complexity of behaviour they produced.

\begin{rigour}[Machina Speculatrix Design]
Each tortoise contained only two sensory inputs and two motors:
\begin{itemize}
    \item \textbf{Photocell}: detected light intensity
    \item \textbf{Bump sensor}: detected physical contact
    \item \textbf{Steering motor}: controlled direction
    \item \textbf{Drive motor}: controlled forward/backward movement
\end{itemize}

The control logic was equally minimal: seek moderate light (avoiding both darkness and bright light), and respond to obstacles by backing up and turning. No explicit programming of ``behaviour'' was included.
\end{rigour}

Despite this minimal design, the tortoises exhibited behaviours that appeared purposeful and even lifelike:

\begin{itemize}
    \item \textbf{Exploration}: wandering through the environment, apparently ``curious''
    \item \textbf{Phototaxis}: moving toward light sources, then circling around them
    \item \textbf{Obstacle avoidance}: navigating around barriers without explicit mapping
    \item \textbf{Self-recognition}: when placed in front of a mirror with a light on their ``nose'', they would exhibit a distinctive ``dance'' as they responded to their own reflection
    \item \textbf{Social behaviour}: when multiple tortoises interacted, they appeared to ``recognise'' each other and engage in complex group dynamics
\end{itemize}

\begin{keybox}[Emergent Behaviour]
\textbf{Emergent behaviour} occurs when a system exhibits properties or patterns that are not explicitly programmed or present in any individual component, but arise from the interactions between components and their environment.

Walter's tortoises demonstrated that complex, apparently intelligent behaviour can emerge from simple rules combined with environmental feedback-no central ``controller'' or explicit behavioural programming required.
\end{keybox}

\subsection{Implications for Complex Systems}

Walter's work established several principles that remain relevant to understanding complex adaptive systems:

\begin{enumerate}
    \item \textbf{Feedback over planning}: The tortoises had no ``plan'' for how to behave; their behaviour emerged from continuous feedback loops with the environment.

    \item \textbf{Simple rules, complex outcomes}: Minimal components and simple interaction rules can produce sophisticated, adaptive behaviour.

    \item \textbf{Environment as computation}: Much of the ``intelligence'' in the system was distributed between the robot and its environment, not located solely in the robot's ``brain''.

    \item \textbf{Robustness through adaptation}: The tortoises could handle novel situations not because they had been programmed for every eventuality, but because their feedback mechanisms allowed continuous adaptation.
\end{enumerate}

These principles directly inform our understanding of effective software development methodologies. Just as Walter's tortoises achieved complex behaviour through simple feedback mechanisms rather than elaborate pre-programming, modern agile methodologies achieve project success through iterative feedback rather than comprehensive upfront planning.


\section{Scrum: Harnessing Complexity}

Scrum is a lightweight framework for managing complex work, particularly software development. Rather than attempting to predict and plan every aspect of a project upfront, Scrum embraces uncertainty and uses rapid feedback cycles to adapt continuously.

\textit{``This concept of a harness to help coordinate independent processors via feedback loops, while having the feedback be reality-based from real data coming from the environment is central to human groups achieving higher level behavior than any individual can achieve on their own.''}

This quote captures the essence of Scrum's philosophy: like Walter's tortoises, a Scrum team achieves outcomes beyond what any individual member could accomplish alone, not through rigid command-and-control hierarchies, but through structured feedback mechanisms that enable emergent coordination.

\subsection{Historical Context}

The term ``Scrum'' comes from rugby, where it refers to the method of restarting play after an infringement-a tight formation where the team works together to move the ball forward. The metaphor emphasises teamwork, adaptability, and collective effort toward a shared goal.

Scrum was formalised in the mid-1990s by Jeff Sutherland and Ken Schwaber, drawing on earlier work in lean manufacturing (particularly the Toyota Production System), empirical process control theory, and iterative development practices. The framework was designed to address the chronic problems of traditional ``waterfall'' development:

\begin{itemize}
    \item Projects delivered late and over budget
    \item Final products that didn't meet actual user needs
    \item Inability to respond to changing requirements
    \item Developer burnout and disengagement
    \item Integration nightmares when combining work from multiple teams
\end{itemize}

\subsection{Scrum Principles}

\begin{keybox}[Core Scrum Principles]
\begin{itemize}
    \item \textbf{Inspect and adapt}: Continuously examine outcomes and adjust processes accordingly
    \item \textbf{Embrace uncertainty}: Accept that surprises will occur (and that this is valuable information, not failure)
    \item \textbf{Short development cycles}: Keep iterations brief (typically 1--4 weeks) to enable rapid feedback
    \item \textbf{Small, achievable goals}: Break work into discrete, completable chunks rather than monolithic deliverables
    \item \textbf{Frequent re-evaluation}: Regularly reassess priorities based on new information
    \item \textbf{Regular check-ins}: Maintain team alignment through structured communication rituals
    \item \textbf{Transparency}: Make work visible to all stakeholders
    \item \textbf{Self-organisation}: Teams determine how to accomplish work, not managers
\end{itemize}
\end{keybox}

\begin{redbox}[Common Misconception]
Scrum is not ``no planning''-it is \textit{continuous planning}. Traditional waterfall approaches front-load all planning before development begins. Scrum distributes planning throughout the project, with detailed planning happening just before work begins (when information is freshest) and high-level planning being continuously refined based on empirical evidence.
\end{redbox}


\section{The Agile Manifesto}

Scrum exists within the broader context of the Agile movement. In February 2001, seventeen software developers met at a ski resort in Utah and produced the \textit{Manifesto for Agile Software Development}, a brief document that articulated shared values underlying their various methodologies.

\begin{rigour}[The Agile Manifesto]
\textbf{We are uncovering better ways of developing software by doing it and helping others do it. Through this work we have come to value:}

\begin{center}
\begin{tabular}{rcl}
\textbf{Individuals and interactions} & over & processes and tools \\
\textbf{Working software} & over & comprehensive documentation \\
\textbf{Customer collaboration} & over & contract negotiation \\
\textbf{Responding to change} & over & following a plan \\
\end{tabular}
\end{center}

\textbf{That is, while there is value in the items on the right, we value the items on the left more.}
\end{rigour}

Note the careful phrasing: the manifesto does not say that processes, documentation, contracts, and plans are worthless. It establishes a hierarchy of values-when trade-offs must be made, prefer the items on the left.

\subsection{The Twelve Principles}

Behind the manifesto lie twelve principles that elaborate on these values:

\begin{enumerate}
    \item Our highest priority is to satisfy the customer through early and continuous delivery of valuable software.

    \item Welcome changing requirements, even late in development. Agile processes harness change for the customer's competitive advantage.

    \item Deliver working software frequently, from a couple of weeks to a couple of months, with a preference to the shorter timescale.

    \item Business people and developers must work together daily throughout the project.

    \item Build projects around motivated individuals. Give them the environment and support they need, and trust them to get the job done.

    \item The most efficient and effective method of conveying information to and within a development team is face-to-face conversation.

    \item Working software is the primary measure of progress.

    \item Agile processes promote sustainable development. The sponsors, developers, and users should be able to maintain a constant pace indefinitely.

    \item Continuous attention to technical excellence and good design enhances agility.

    \item Simplicity-the art of maximising the amount of work not done-is essential.

    \item The best architectures, requirements, and designs emerge from self-organising teams.

    \item At regular intervals, the team reflects on how to become more effective, then tunes and adjusts its behaviour accordingly.
\end{enumerate}

\begin{keybox}[Agile vs Scrum]
\textbf{Agile} is a philosophy and set of values about how software should be developed.

\textbf{Scrum} is a specific framework that implements agile principles through defined roles, events, and artefacts.

Other agile methodologies include Kanban, Extreme Programming (XP), Crystal, and Feature-Driven Development. Scrum is the most widely adopted, but it is not synonymous with Agile.
\end{keybox}


\section{The Scrum Framework}

Scrum provides a minimal but complete framework for managing iterative development. It consists of three categories of elements: roles (who), events (when), and artefacts (what).

\subsection{Scrum Roles}

A Scrum team consists of exactly three roles. This constraint is intentional-it prevents the proliferation of specialised roles that can create bottlenecks and diffuse accountability.

\subsubsection{Product Owner}

The Product Owner is responsible for maximising the value of the product and the work of the Development Team. Key responsibilities include:

\begin{itemize}
    \item \textbf{Managing the Product Backlog}: Maintaining the prioritised list of features, enhancements, and fixes
    \item \textbf{Expressing backlog items clearly}: Ensuring the team understands what needs to be built and why
    \item \textbf{Prioritising work}: Ordering the backlog to optimise value delivery
    \item \textbf{Stakeholder communication}: Serving as the primary interface between the team and external stakeholders
    \item \textbf{Acceptance decisions}: Determining whether work meets the definition of ``done''
\end{itemize}

The Product Owner must be a single individual (not a committee) with sufficient authority to make decisions about the product. They represent the voice of the customer and the business.

\subsubsection{Scrum Master}

The Scrum Master is responsible for ensuring Scrum is understood and enacted. They serve the team by:

\begin{itemize}
    \item \textbf{Facilitating Scrum events}: Ensuring meetings happen and are productive
    \item \textbf{Removing impediments}: Clearing obstacles that slow the team's progress
    \item \textbf{Coaching the team}: Helping team members understand and apply Scrum effectively
    \item \textbf{Protecting the team}: Shielding the team from external interruptions and scope creep during sprints
    \item \textbf{Promoting self-organisation}: Helping the team develop its ability to manage itself
\end{itemize}

The Scrum Master is explicitly \textit{not} a traditional project manager. They have no authority to assign tasks or make decisions for the team. Their role is servant-leadership: enabling the team's success rather than directing their work.

\subsubsection{Development Team}

The Development Team consists of professionals who do the work of delivering a potentially releasable increment of product at the end of each Sprint. Characteristics include:

\begin{itemize}
    \item \textbf{Self-organising}: The team decides how to accomplish work; no one tells them how to turn backlog items into increments of functionality
    \item \textbf{Cross-functional}: The team collectively possesses all skills needed to create the product increment
    \item \textbf{No titles}: Regardless of individual specialisations, all members are ``Developers''
    \item \textbf{No sub-teams}: Work is not siloed; the whole team is accountable for the whole increment
    \item \textbf{Optimal size}: Typically 3--9 people (large enough for diverse skills, small enough for effective coordination)
\end{itemize}

\begin{rigour}[Team Accountability]
In Scrum, accountability operates at the team level, not the individual level. If the team fails to deliver the sprint goal, the \textit{team} failed-not any individual member. This encourages collaboration over hero culture and creates psychological safety for experimentation and learning.
\end{rigour}


\subsection{Scrum Events}

Scrum prescribes five events (sometimes called ``ceremonies''). Each creates a formal opportunity for inspection and adaptation. All events are time-boxed-they have a maximum duration that cannot be extended.

\subsubsection{The Sprint}

The Sprint is the heartbeat of Scrum-a time-boxed iteration (typically 1--4 weeks, with 2 weeks being most common) during which a ``Done'', usable, potentially releasable product increment is created.

\begin{keybox}[Sprint Characteristics]
\begin{itemize}
    \item \textbf{Fixed duration}: Sprints are always the same length; consistency aids planning and measurement
    \item \textbf{Immutable goal}: Once the sprint begins, the Sprint Goal cannot be changed
    \item \textbf{Protected scope}: No changes that endanger the Sprint Goal are allowed during the sprint
    \item \textbf{Continuous}: A new sprint starts immediately after the previous one ends
    \item \textbf{Cancellable}: Only the Product Owner can cancel a sprint, and only if the Sprint Goal becomes obsolete
\end{itemize}
\end{keybox}

\subsubsection{Sprint Planning}

Sprint Planning initiates the sprint by defining what can be delivered and how. The entire Scrum team collaborates. For a two-week sprint, this meeting is time-boxed to four hours maximum.

The meeting addresses two questions:

\begin{enumerate}
    \item \textbf{What can be done this sprint?} The Product Owner presents the highest-priority backlog items. The Development Team forecasts how much they can accomplish based on their past performance (velocity) and current capacity.

    \item \textbf{How will the work get done?} The Development Team decomposes selected backlog items into tasks, typically of one day or less. This creates the Sprint Backlog.
\end{enumerate}

By the end of Sprint Planning, the team has a Sprint Goal-a coherent objective that provides guidance on why the increment is being built.

\subsubsection{Daily Scrum}

The Daily Scrum (often called ``standup'' because participants traditionally stand to keep the meeting brief) is a 15-minute time-boxed event for the Development Team to synchronise and plan the next 24 hours.

Each team member typically addresses three questions:
\begin{enumerate}
    \item What did I accomplish yesterday toward the Sprint Goal?
    \item What will I work on today toward the Sprint Goal?
    \item Are there any impediments blocking my progress?
\end{enumerate}

\begin{redbox}[Daily Scrum Anti-patterns]
The Daily Scrum is \textit{not}:
\begin{itemize}
    \item A status report to management
    \item A problem-solving session (those happen after the standup)
    \item An opportunity to assign or reassign tasks
    \item A meeting that requires the Scrum Master or Product Owner to run (the Development Team owns it)
\end{itemize}

If the Daily Scrum regularly runs over 15 minutes, or if team members are speaking to the Scrum Master rather than to each other, these are signals that the meeting has drifted from its purpose.
\end{redbox}

\subsubsection{Sprint Review}

The Sprint Review is held at the end of the Sprint to inspect the Increment and adapt the Product Backlog if needed. The team demonstrates what was accomplished, and stakeholders provide feedback. For a two-week sprint, this is time-boxed to two hours.

Key activities include:
\begin{itemize}
    \item The Product Owner explains what has been ``Done'' and what has not
    \item The Development Team demonstrates the work and answers questions
    \item The Product Owner discusses the current Product Backlog state
    \item The group collaborates on what to do next
    \item The meeting results in a revised Product Backlog for the next Sprint
\end{itemize}

\subsubsection{Sprint Retrospective}

The Sprint Retrospective is an opportunity for the Scrum Team to inspect itself and create a plan for improvements. It occurs after the Sprint Review and before the next Sprint Planning. For a two-week sprint, this is time-boxed to 90 minutes.

The team discusses:
\begin{enumerate}
    \item What went well during the Sprint?
    \item What could be improved?
    \item What will we commit to improving in the next Sprint?
\end{enumerate}

\begin{keybox}[Continuous Improvement]
The Retrospective is where the ``inspect and adapt'' principle applies to the team's own processes, not just the product. This institutionalised reflection is one of Scrum's most powerful features-it builds continuous improvement into the team's operating rhythm.

Teams that skip or rush retrospectives often find their velocity plateaus. The retrospective is an investment in the team's future effectiveness.
\end{keybox}


\subsection{Scrum Artefacts}

Scrum's artefacts represent work or value. They are designed to maximise transparency of key information.

\subsubsection{Product Backlog}

The Product Backlog is an ordered list of everything that is known to be needed in the product. It is the single source of requirements for any changes to be made to the product.

Characteristics:
\begin{itemize}
    \item \textbf{Dynamic}: The backlog constantly evolves as the product and its environment change
    \item \textbf{Never complete}: As long as a product exists, its backlog exists
    \item \textbf{Ordered by value}: Higher-priority items are typically more detailed and refined
    \item \textbf{Owned by Product Owner}: The Product Owner is responsible for content, availability, and ordering
\end{itemize}

Product Backlog items typically include features, functions, requirements, enhancements, and fixes. Each item has a description, order, estimate, and value. Higher-order items are clearer and more detailed; lower-order items may be deliberately vague until they move up the backlog.

\subsubsection{Sprint Backlog}

The Sprint Backlog is the set of Product Backlog items selected for the Sprint, plus a plan for delivering them and achieving the Sprint Goal. It makes visible all the work the Development Team identifies as necessary to meet the Sprint Goal.

The Sprint Backlog is a forecast by the Development Team about what functionality will be in the next Increment and the work needed to deliver that functionality. It is owned entirely by the Development Team and is updated throughout the Sprint as work is completed and new work is discovered.

\subsubsection{Product Increment}

The Increment is the sum of all Product Backlog items completed during a Sprint and all previous Sprints. At the end of a Sprint, the new Increment must be ``Done''-meaning it is in a usable condition and meets the team's Definition of Done.

\begin{rigour}[Definition of Done]
The Definition of Done (DoD) is a shared understanding of what it means for work to be complete. It typically includes criteria such as:
\begin{itemize}
    \item Code is written and reviewed
    \item Unit tests pass
    \item Integration tests pass
    \item Documentation is updated
    \item Code is merged to the main branch
    \item Product Owner has accepted the work
\end{itemize}

The DoD creates transparency by ensuring everyone has a shared understanding of ``complete''. Without an explicit DoD, ``done'' becomes subjective and technical debt accumulates.
\end{rigour}


\section{Why This Matters for Data Structures and Algorithms}

You might wonder why a course on data structures and algorithms begins with software development methodology. The connection is threefold:

\subsection{Algorithms Exist in Context}

Algorithms do not exist in a vacuum. They are implemented within software systems, developed by teams, and deployed to solve real problems. Understanding how professional software is developed helps you:

\begin{itemize}
    \item Write code that others can understand and maintain
    \item Break complex algorithmic problems into manageable pieces
    \item Estimate how long implementation will take
    \item Collaborate effectively with other developers
\end{itemize}

\subsection{Iterative Problem-Solving}

The ``inspect and adapt'' philosophy of Scrum mirrors effective algorithmic problem-solving:

\begin{enumerate}
    \item Start with a simple, potentially inefficient solution
    \item Test it against your understanding of the problem
    \item Identify bottlenecks and areas for improvement
    \item Refine the solution based on empirical evidence
    \item Repeat until the solution meets requirements
\end{enumerate}

This iterative approach-rather than attempting to design a perfect algorithm from scratch-is how experienced practitioners actually solve algorithmic problems.

\subsection{Managing Complexity}

Both Scrum and algorithm design are fundamentally about managing complexity:

\begin{itemize}
    \item \textbf{Decomposition}: Breaking complex problems into smaller, tractable sub-problems
    \item \textbf{Abstraction}: Hiding unnecessary detail behind clean interfaces
    \item \textbf{Empiricism}: Making decisions based on measured evidence rather than speculation
    \item \textbf{Feedback loops}: Using results to inform future decisions
\end{itemize}

Grey Walter's tortoises achieved complex behaviour through simple feedback mechanisms. Scrum teams achieve complex software through structured iteration. And you will achieve elegant algorithmic solutions through the same principles: break the problem down, implement something, measure it, and refine.

\begin{keybox}[The Unifying Theme]
Whether designing autonomous robots, managing software projects, or implementing algorithms, the same insight applies: \textbf{complex adaptive behaviour emerges from simple rules combined with feedback from reality}.

This course will teach you the vocabulary and techniques for analysing algorithms. But the meta-skill-iterative refinement through empirical feedback-transcends any particular algorithmic technique.
\end{keybox}
